\documentclass[a4paper,10pt]{article}
\usepackage[utf8]{inputenc}
\usepackage[margin=1in]{geometry}
\usepackage{chemformula}
\usepackage{amsmath}
\usepackage[table,xcdraw]{}
\usepackage{colortbl}
\usepackage{wrapfig}



%\section{Disidratazione del saccarosio}
%\subsection{Principi teorici}
%\subsection{Scopo}
%\subsection{Materiali}
%\subsection{Reagenti}
%\subsection{Procedimento}
%\subsection{Risultati}






%opening
\title{\vspace{-4em}{\large Liceo Scientifico T.L. Caro} \\
	Relazione di Laboratorio}
\author{
	Emanuele De Filippo, Augusto Strianese, Simone Esposito, \\
	Mattia Carbone, Massimo De Luca
}
\date{8 Febbraio 2024}
\begin{document}
	
	\maketitle
	
	
	\section{Esperimento primo}
	\subsection{Principi teorici}
	\begin{itemize}
		\item Legge di Dalton.
		\item Legge di Proust.
		\item Legge di Lavoisier.
	\end{itemize}
	\subsection{Scopo}
	
	Riconoscere una trasformazione chimica irreversibile nella carbonizzazione del saccarosio.
	
	\subsection{Materiali}
	\begin{itemize}
		\item{Becher in vetro pirex da 250ml.}
		\item{Cilindro volumetrico da 100ml.}
		\item Camice, Guanti e occhiali protettivi.
	\end{itemize}
	
	
	\subsection{Reagenti}
	\begin{itemize}
		\item{Acido Solforico \ch{H2SO4} 98\%.}
		\item{Saccarosio} \ch{C12H22O11} 20g.
	\end{itemize}
	
	
	\subsection{Procedimento}
	\begin{enumerate}
		\item Sotto una cappa da aspirazione, versare 25ml di Acido solforico all'interno di un cilindro volumetrico.
		\item Introdurre 20g di saccarosio in un becher.
		\item Versare lentamente l'acido solforico dal cilindro dentro al becher.
	\end{enumerate}
	\subsection{Risultati}
	\begin{itemize}
		\item Dopo alcuni minuti la sostanza inizia a scurirsi data la reazione del saccarosio che si decompone in carbonio (che da il colore nero) e acqua secondo reazione chimica (1).
		\begin{equation}
			\ch{C12H22O11 ->[ H2SO4 ] 12 C + 11 H2O}
		\end{equation}
		
		\item Durante la reazione si e liberato \ch{SO4} gassoso secondo la reazione chimica (2).
		\begin{equation}
			\ch{H2SO4 -> H2O + SO4}
		\end{equation}
		\item La reazione è esotermica, e se mettiamo il palmo della mano sotto la          base del becher si nota che è tiepido.
		\item Il saccarosio ha assunto una consistenza pastosa e non è in grado di tornare allo stato precedente.
	\end{itemize}
	
	
	
	\par\noindent\rule{\textwidth}{0.5pt}
	
	
	
	\section{Esperimento secondo}
	\subsection{Principi teorici}
	\begin{itemize}
		\item Legge di Lavoisier.
		\item Legge di Dalton.
		\item Legge di Proust.
	\end{itemize}
	\subsection{Scopo}
	Notare il cambio di pH della soluzione attraverso la carta al tornasole.
	\subsection{Materiali}
	\begin{itemize}
		\item Becher in vetro pirex da 50ml.
		\item Cannucce in vetro.
		\item Carta al tornasole.
		\item Spatole.
	\end{itemize}
	\subsection{Reagenti}
	\begin{itemize}
		\item Idrossido di Calcio.
		\item Acqua distillata.
	\end{itemize}
	\subsection{Procedimento}
	\begin{enumerate}
		\item Sciogliere una punta di spatola di idrossido di calcio in 30ml di acqua distillata.
		\item Nel becher, inserire una striscia di carta al tornasole.
		\item Nel becher, usare la cannuccia per soffiare al suo interno, quindi immetendo anidride carbonica nella soluzione che reagirà con l'idrossido di calcio per formare l'acido carbonico secondo la reazione (3), che alzerà il pH della soluzione. E formerà carbonato di calcio secondo la reazione (4) che precipiterà.
		\begin{equation}
			\ch{H2O + CO2 <=> H2CO3}
		\end{equation}
		\begin{equation}
			\ch{Ca(OH)2 + CO2 -> CaCO3 + H2O}
		\end{equation}
		
	\end{enumerate}
	\subsection{Risultati}
	La soluzione inizialmente sarà basica data la presenza del idrossido di sodio, ma dopo avervi introdotto anidride carbonica, avverrà la produzione di acido carbonico che bilancerà il pH della soluzione e cambierà il colore della carta al tornasole. Il carbonato di calcio invece precipiterà alla base del becher.
	
	\break
	
	
	
	
	
	
	
	
	
	
	
	\section{Esperimento terzo}
	\subsection{Principi teorici}
	\begin{itemize}
		\item Legge di Lavoisier.
		\item Legge di Proust.
		\item Legge di Dalton.
	\end{itemize}
	\subsection{Scopo}
	Notare i cambiamenti dei colori e riconoscere i precipitati e i surnatanti.
	\subsection{Materiali}
	\begin{itemize}
		\item Provetta.
		\item Spatola.
		\item Becco Bunsen.
		\item Ancoretta magnetica.
		\item Agitatore magnetico.
		\item Ancoretta magnetica.
		\item Navicella.
		\item Bilancia.
	\end{itemize}
	\subsection{Reagenti}
	\begin{itemize}
		\item Acqua distillata \ch{H2O}.
		\item Idrossido di sodio \ch{NaOH}.
		\item Solfato di rame \ch{{CuSO4} 5 H2O}.
	\end{itemize}
	\subsection{Procedimento}
	\begin{enumerate}
		\item Scogliere una punta di spatola di rame pentaidrato in acqua distillata dentro un becher.
		\item Porre il becher su un agitatore magnetico e osservare il cambiamento di colore della soluzione (azzurrino).
		\item spostare la soluzione di solfato di rame pentaidrato in una provetta.
		\item Calcolare i grammi di idrossido di sodio 2 molare.
		
		
		\begin{math}
			M = 2M \\
			V = 0.2L \\ 
			MM_{\ch{NaOH}} = 40 \ \frac{\text{g}}{\text{mol}} \\
			M = \frac{n}{V} \implies n = MV \\
			n_{\ch{NaOH}} = MV = 2 \times 0.2 = 0.4 \ \text{mol} \\
			n = \frac{m}{MM} \implies m = n MM \\
			m_{\ch{NaOH}} = n_{\ch{NaOH}} MM_{\ch{NaOH}} = 0.4 \times 40 = 1.6  \ \text{g}
		\end{math}
		\item Sulla bilancia, pesare nella navicella 16g di idrossido di sodio aiutandosi con una spatola.
		\item In un becher da 250ml, introdurre 200ml di acqua distillata, 16g di idrossido di sodio e un ancoretta magnetica.
		\item Porre il becher con la soluzione di idrossido di sodio 2M sull'agitatore magnetico.
		\item Utilizzando una pipetta, aggiungere gocce della soluzione di idrossido di sodio 2M alla provetta con la soluzione di solfato di rame pentaidrato innescando una reazione che forma idrossido di rame secondo la reazione (5).
		
		\begin{equation}
			\ch{CuSO4 + 2 NaOH -> Cu(OH)2 + Na2SO4}
		\end{equation}
		\item Accendere il becco Bunsen.
		\item Con una pinza, prendere la provetta e porla sulla fiamma del Bunsen (facendo attenzione a non far fuoriuscire la soluzione dalla provetta) avviando quindi una reazione di decomposizione (6).
		\begin{equation}
			\ch{Cu(OH)2 -> [$\Delta$] CuO + H2O}
		\end{equation}
		
	\end{enumerate}
	\subsection{Risultati}
	\begin{itemize}
		\item Quando abbiamo sciolto il solfato di rame pentaidrato all interno dell'acqua distillata la soluzione ha assunto un colore azzurrino.
		\item Durante la reazione (5) si è notato un precipitato blu, l'idrossido di rame.
		\item Durante la reazione (6) si sono osservati la separazione del surnatante (acqua) e del precipitante (ossido di rame).
	\end{itemize}
	\break
	
	
	
	
	
	\section{Esperimento quarto}
	\subsection{Principi teorici}
	\begin{itemize}
		\item Legge di Proust.
		\item Legge di Dalton.
		\item Legge di Lavoiser.
	\end{itemize}
	
	\subsection{Scopo}
	Riconoscere la reazione fra acido cloridrico e ferro attraverso lo sviluppo di un gas.
	\subsection{Materiali}
	\begin{itemize}
		\item Becher da 50ml.
		\item Becher da 80ml.
		\item Pipetta.
	\end{itemize}
	\subsection{Reagenti}
	\begin{itemize}
		\item Limatura di ferro.
		\item Acido cloridrico \ce{HCl} in soluzione $1 : 3$.
	\end{itemize}
	\subsection{Procedimento}
	\begin{itemize}
		\item Introdurre una spatolata di limatura di ferro in un becher da 80ml.
		\item Riempire un becher da 50ml con 25ml di soluzione di acido cloridrico.
		\item Versare l'intero contenuto del becher da 50ml in una beuta.
		\item Attraverso una pipetta, aggiungere la soluzione di acido cloridrico nel becher da 80ml, innescando una reazione fra ferro e acido cloridrico che produrrà cloruro di ferro e idrogeno secondo la reazione (7).
		\begin{equation}
			\ch{Fe + 2 HCl -> FeCl2 + H2}
		\end{equation}
	\end{itemize}
	\subsection{Risultati}
	\begin{itemize}
		\item Come riportato nella reazione (7), ferro e acido cloridrico reagiscono per formare cloruro di ferro e idrogeno, mentre il cloruro di ferro dona il colore giallo alla soluzione, l'idrogeno si può notare sotto forma di bollicine che possono essere incendiate.
	\end{itemize}
	
	\break
	
	\section{Saggi alla fiamma}
	\subsection{Principi teorici}
	\begin{itemize}
		\item Legge di Planck.
		\item Livelli energetici.
	\end{itemize}
	\subsection{Scopo}
	Osservare i salti quantici degli elettroni eccitati, attraverso le emissioni di energia attraverso fotoni visibili come raggi luminosi.
	\subsection{Materiali}
	\begin{itemize}
		\item 2 Becher in vetro pirex, uno da 100ml e uno da 250ml.
		\item Becco Bunsen.
		\item 5 Spatole.
		\item Filo di nickelcromo.
		\item 5 Navette in plastica.
	\end{itemize}
	\subsection{Reagenti}
	\begin{itemize}
		\item Acido cloridrico \ch{HCl} 98\%.
		\item Acqua distillata \ch{H2O}.
		\item Cloruro di stronzio \ch{SrCl2}.
		\item Cloruro di potassio \ch{KCl}.
		\item Solfato di magnesio \ch{MgSO4}.
		\item Solfato di rame pentaidrato \ch{CuSO4 5 H2O}.
		\item Cloruro di sodio \ch{NaCl}.
	\end{itemize}
	\subsection{Procedimento}
	\begin{enumerate}
		\item Introdurre 50ml di acido cloridrico ad un becher da 100ml.
		\item Introdurre 100ml di acqua distillata in un becher da 250ml.
		\item Versare l'acido cloridrico lentamente nell'acqua distillata.
		\item Accendere il becco Bunsen.
		\item Inumidire il filo di nickelcromo nel acido diluito.
		\item Raccogliere con il filo di nickelcromo il sale scelto fra i 5 sali recuperati.
		\item Passare con il filo di nickelcromo per la parte ossidante della fiamma.
		\item Se si vuole ripetere il procedimento con un sale differente, assicurarsi, prima di raccogliere un altro sale, di bruciare tutto il sale presente sul filo e, con l'aiuto di uno straccio eliminare ossidi residui, per poi raccogliere un altro sale.
	\end{enumerate}
	\break
	\subsection{Risultati}
	
	Durante la bruciatura dei vari sali noteremo un cambiamento del colore della fiamma, 
	\begin{table}[ht]
		\arrayrulecolor[HTML]{DB5800}
		\centering
		\begin{tabular}{|l|l|}
			\hline
			Sale                & Colore \\ \hline
			\rowcolor[HTML]{FD6864} 
			Cloruro di stronzio & Rosso  \\ \hline
			\rowcolor[HTML]{9698ED} 
			Cloruro di potassio & Lilla  \\ \hline
			Solfato di Magnesio & Bianco \\ \hline
			\rowcolor[HTML]{67FD9A} 
			Solfato di rame     & Verde  \\ \hline
			\rowcolor[HTML]{FFFE65} 
			Cloruro di Sodio    & Giallo \\ \hline
		\end{tabular}
		\label{table:ta}
	\end{table}
	questo cambiamento sarà dovuto ai salti quantici degli elettroni, che inizialmente assorbiranno energia e saliranno da un livello energetico a minor energia ad uno a maggiore energia, successivamente passeranno dal livello a maggior energia allo stato fondamenta e emetteranno energia sotto forma di fotoni, e il colore varierà in base alla frequenza secondo la legge di Planck (8).
	\begin{equation}
		E = h \nu
	\end{equation}
	\par\noindent\rule{\textwidth}{0.5pt}
	
	\break
	
	
	
	
	
	
	
	
	
	
	
\end{document}
